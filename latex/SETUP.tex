% В этом файле следует указать информацию о себе
% и выполняемой работе.

\documentclass [fontsize=14pt, paper=a4, pagesize, DIV=calc]%
{scrreprt}
% ВНИМАНИЕ! Для использования глав поменять
% scrartcl на scrreprt

% Здесь ничего не менять
\usepackage [T2A] {fontenc}   % Кириллица в PDF файле
\usepackage [utf8] {inputenc} % Кодировка текста: utf-8
\usepackage [russian] {babel} % Переносы, лигатуры

%%%%%%%%%%%%%%%%%%%%%%%%%%%%%%%%%%%%%%%%%%%%%%%%%%%%%%%%%%%%%%%%%%%%%%%%
% Создание макроса управления элементами, специфичными
% для вида работы (курс., бак., маг.)
% Здесь ничего не менять:
\usepackage{ifthen}
\newcounter{worktype}
\newcommand{\typeOfWork}[1]
{
	\setcounter{worktype}{#1}
}

% ВНИМАНИЕ!
% Укажите тип работы: 0 - курсовая, 1 - бак., 2 - маг.,
% 3 - бакалаврская с главами.
\typeOfWork{1}
% Считается, что курсовая и бак. бьются на разделы (section) и
% подразделы (subsection), а маг. — на главы (chapter), разделы и
% подразделы. Если хочется,
% чтобы бак. была с главами (например, если она большая),
% надо выбрать опцию 3.

% Если при выборе 2 или 3 вы забудете поменять класс
% документа на scrreprt (см. выше, в самом начале),
% то получите ошибку:
% ./aux/appearance.tex:52: Package scrbase Error: unknown option ` chapterprefix=

%%%%%%%%%%%%%%%%%%%%%%%%%%%%%%%%%%%%%%%%%%%%%%%%%%%%%%%%%%%%%%%%%%%%%%%%
% Информация об авторе и работе для титульной страницы

\usepackage {titling}

% Имя автора в именительном падеже (для маг.)
\newcommand {\me}{%
	А.\,А.~Елизарова%
}

% Имя автора в родительном падеже (для курсовой и бак.)
\newcommand {\byme}{%
	А.\,А.~Елизаровой%
}

% Научный руководитель
\newcommand{\supervisor}%
{доктор физ.-мат. наук В.А. Скороходов}

% идентифицируем пол (только для курсовой и бак.)
\newcommand{\bystudent}{
	студентки %Студента % Для курсовой: с большой буквы
}

% Год публикации
\date{2019}

% Название работы
\title{Случайные процессы на графах с особыми видами нестандартной достижимости}

% Кафедра
%

\newcommand {\direction} {%
	Направление подготовки\\01.\ifthenelse{\value{worktype} = 2}{04}{03}.02 ---
	Прикладная математика\\и информатика%
}

%%%%%%%%%%%%%%%%%%%%%%%%%%%%%%%%%%%%%%%%%%%%%%%%%%%%%%%%%%%%%%%%%%%%%%%%
% Другие настраиваемые элементы текста

% Листинги с исходным кодом программ: укажите язык программирования
\usepackage{listings}
\lstset{
	language=Python,%  Язык указать здесь
	basicstyle=\small\ttfamily,
	breaklines=true,%
	showstringspaces=false%
	inputencoding=utf8x%
	extendedchars=true,
	literate={Ö}{{\"O}}1
	{Ä}{{\"A}}1
	{Ü}{{\"U}}1
	{ß}{{\ss}}1
	{ü}{{\"u}}1
	{ä}{{\"a}}1
	{ö}{{\"o}}1
	{~}{{\textasciitilde}}1
	{а}{{\selectfont\char224}}1
	{б}{{\selectfont\char225}}1
	{в}{{\selectfont\char226}}1
	{г}{{\selectfont\char227}}1
	{д}{{\selectfont\char228}}1
	{е}{{\selectfont\char229}}1
	{ё}{{\"e}}1
	{ж}{{\selectfont\char230}}1
	{з}{{\selectfont\char231}}1
	{и}{{\selectfont\char232}}1
	{й}{{\selectfont\char233}}1
	{к}{{\selectfont\char234}}1
	{л}{{\selectfont\char235}}1
	{м}{{\selectfont\char236}}1
	{н}{{\selectfont\char237}}1
	{о}{{\selectfont\char238}}1
	{п}{{\selectfont\char239}}1
	{р}{{\selectfont\char240}}1
	{с}{{\selectfont\char241}}1
	{т}{{\selectfont\char242}}1
	{у}{{\selectfont\char243}}1
	{ф}{{\selectfont\char244}}1
	{х}{{\selectfont\char245}}1
	{ц}{{\selectfont\char246}}1
	{ч}{{\selectfont\char247}}1
	{ш}{{\selectfont\char248}}1
	{щ}{{\selectfont\char249}}1
	{ъ}{{\selectfont\char250}}1
	{ы}{{\selectfont\char251}}1
	{ь}{{\selectfont\char252}}1
	{э}{{\selectfont\char253}}1
	{ю}{{\selectfont\char254}}1
	{я}{{\selectfont\char255}}1
	{А}{{\selectfont\char192}}1
	{Б}{{\selectfont\char193}}1
	{В}{{\selectfont\char194}}1
	{Г}{{\selectfont\char195}}1
	{Д}{{\selectfont\char196}}1
	{Е}{{\selectfont\char197}}1
	{Ё}{{\"E}}1
	{Ж}{{\selectfont\char198}}1
	{З}{{\selectfont\char199}}1
	{И}{{\selectfont\char200}}1
	{Й}{{\selectfont\char201}}1
	{К}{{\selectfont\char202}}1
	{Л}{{\selectfont\char203}}1
	{М}{{\selectfont\char204}}1
	{Н}{{\selectfont\char205}}1
	{О}{{\selectfont\char206}}1
	{П}{{\selectfont\char207}}1
	{Р}{{\selectfont\char208}}1
	{С}{{\selectfont\char209}}1
	{Т}{{\selectfont\char210}}1
	{У}{{\selectfont\char211}}1
	{Ф}{{\selectfont\char212}}1
	{Х}{{\selectfont\char213}}1
	{Ц}{{\selectfont\char214}}1
	{Ч}{{\selectfont\char215}}1
	{Ш}{{\selectfont\char216}}1
	{Щ}{{\selectfont\char217}}1
	{Ъ}{{\selectfont\char218}}1
	{Ы}{{\selectfont\char219}}1
	{Ь}{{\selectfont\char220}}1
	{Э}{{\selectfont\char221}}1
	{Ю}{{\selectfont\char222}}1
	{Я}{{\selectfont\char223}}1
	{і}{{\selectfont\char105}}1
	{ї}{{\selectfont\char168}}1
	{є}{{\selectfont\char185}}1
	{ґ}{{\selectfont\char160}}1
	{І}{{\selectfont\char73}}1
	{Ї}{{\selectfont\char136}}1
	{Є}{{\selectfont\char153}}1
	{Ґ}{{\selectfont\char128}}1
}

% полный список языков, поддерживаемых данным пакетом, есть,
% например, здесь (стр. 13):
% ftp://ftp.tex.ac.uk/tex-archive/macros/latex/contrib/listings/listings.pdf

% Нумерация списков: можно при необходимести
% изменять вид нумерации (например, добавлять правую скобку).
% По умолчанию буду списки вида:
% 1.
% 2.
% Изменять вид нумерации можно в начале нумерации:
% \begin{enumerate}[1)] (В квадратных скобках указан желаемый вид)
\usepackage[shortlabels]{enumitem}
\setlist[enumerate, 1]{1.}

% Гиперссылки: настройте внешний вид ссылок
\usepackage%
[pdftex,unicode,pdfborder={0 0 0},draft=false,%backref=page,
hidelinks, % убрать, если хочется видеть ссылки: это
% удобно в PDF файле, но не должно появиться на печати
bookmarks=true,bookmarksnumbered=false,bookmarksopen=false]%
{hyperref}
