%%%%%%%%%%%%%%%%%%%%%%%%%%%%%%%%%%%%%%%%%
% Beamer Presentation
% LaTeX Template
% Version 1.0 (10/11/12)
%
% This template has been downloaded from:
% http://www.LaTeXTemplates.com
%
% License:
% CC BY-NC-SA 3.0 (http://creativecommons.org/licenses/by-nc-sa/3.0/)
%
%%%%%%%%%%%%%%%%%%%%%%%%%%%%%%%%%%%%%%%%%

%----------------------------------------------------------------------------------------
%	PACKAGES AND THEMES
%----------------------------------------------------------------------------------------

\documentclass{beamer}

% The Beamer class comes with a number of default slide themes
% which change the colors and layouts of slides. Below this is a list
% of all the themes, uncomment each in turn to see what they look like.

%\usetheme{default}
%\usetheme{AnnArbor}
%\usetheme{Antibes}
%\usetheme{Bergen}
%\usetheme{Berkeley}
%\usetheme{Berlin}
%\usetheme{Boadilla}
%\usetheme{CambridgeUS}
%\usetheme{Copenhagen}
%\usetheme{Darmstadt}
%\usetheme{Dresden}
%\usetheme{Frankfurt}
%\usetheme{Goettingen}
%\usetheme{Hannover}
%\usetheme{Ilmenau}
%\usetheme{JuanLesPins}
%\usetheme{Luebeck}
\usetheme{Madrid}
%\usetheme{Malmoe}
%\usetheme{Marburg}
%\usetheme{Montpellier}
%\usetheme{PaloAlto}
%\usetheme{Pittsburgh}
%\usetheme{Rochester}
%\usetheme{Singapore}
%\usetheme{Szeged}
%\usetheme{Warsaw}

% As well as themes, the Beamer class has a number of color themes
% for any slide theme. Uncomment each of these in turn to see how it
% changes the colors of your current slide theme.

%\usecolortheme{albatross}
%\usecolortheme{beaver}
%\usecolortheme{beetle}
%\usecolortheme{crane}
%\usecolortheme{dolphin}
%\usecolortheme{dove}
%\usecolortheme{fly}
%\usecolortheme{lily}
%\usecolortheme{orchid}
%\usecolortheme{rose}
%\usecolortheme{seagull}
%\usecolortheme{seahorse}
%\usecolortheme{whale}
%\usecolortheme{wolverine}

%\setbeamertemplate{footline} % To remove the footer line in all slides uncomment this line
%\setbeamertemplate{footline}[page number] % To replace the footer line in all slides with a simple slide count uncomment this line

%\setbeamertemplate{navigation symbols}{} % To remove the navigation symbols from the bottom of all slides uncomment this line


\usepackage{graphicx} % Allows including images
\usepackage{booktabs} % Allows the use of \toprule, \midrule and \bottomrule in tables

\usepackage[utf8]{inputenc}
\usepackage[T2A]{fontenc}
\usepackage[russian,english]{babel}

\setbeamertemplate{navigation symbols}{} % To remove the navigation symbols from the bottom of all slides uncomment this line

\setbeamercolor{footline}{}
\setbeamertemplate{footline}{
  \leavevmode%
  \hbox{%
  \begin{beamercolorbox}[wd=.333333\paperwidth,ht=2.25ex,dp=1ex,center]{}%
    А. А. Елизарова (Мехмат ЮФУ)
  \end{beamercolorbox}%
  \begin{beamercolorbox}[wd=.333333\paperwidth,ht=2.25ex,dp=1ex,center]{}%
    Ростов-на-Дону, 2019
  \end{beamercolorbox}%
  \begin{beamercolorbox}[wd=.333333\paperwidth,ht=2.25ex,dp=1ex,right]{}%
  Стр. \insertframenumber{} из \inserttotalframenumber \hspace*{2ex}
  \end{beamercolorbox}}%
  \vskip0pt%
}

%----------------------------------------------------------------------------------------
%	TITLE PAGE
%----------------------------------------------------------------------------------------

\title{\small{СЛУЧАЙНЫЕ ПРОЦЕССЫ НА ГРАФАХ С\\ОСОБЫМИ ВИДАМИ НЕСТАНДАРТНОЙ\\ДОСТИЖИМОСТИ}}
\vspace{15pt}%
\author{\small{%
А. А. Елизарова\\%
\emph{Направление подготовки:}~Прикладная математика и\\информатика\\%
\emph{Научный руководитель:}~проф., д.ф.-м.н. В. А. Скороходов}\\%
\vspace{15pt}%
    Южный федеральный университет\\
    Институт математики, механики и компьютерных наук
    имени~И.\,И.\,Воровича%
}
\date{\small{Ростов-на-Дону, 2019}}

\begin{document}

	\begin{frame}
		\titlepage % Print the title page as the first slide
	\end{frame}

	\begin{frame}
		\frametitle{Содержание} % Table of contents slide, comment this block out to remove it
		\tableofcontents % Throughout your presentation, if you choose to use \section{} and \subsection{} commands, these will automatically be printed on this slide as an overview of your presentation
	\end{frame}

%----------------------------------------------------------------------------------------
%	PRESENTATION SLIDES
%----------------------------------------------------------------------------------------

%-----------------------------------------------------------------------------------------
%-----------------------------------------------------------------------------------------
\section{Постановка задачи}
%-----------------------------------------------------------------------------------------

\begin{frame}
\frametitle{Постановка задачи}

	\begin{itemize}
		\item исследовать графы с глобальными условиями на достижимость,
		\item разработать методы решения классических задач на графах с
		глобальными условиями на достижимость,
		\item рассмотреть задачу о случайном блуждании частиц на графах с глобальными условиями на достижимость,
		\item предложить методы нахождения предельного распределения для
		исследуемого процесса случайного блуждания.
	\end{itemize}

\end{frame}

%-----------------------------------------------------------------------------------------
%-----------------------------------------------------------------------------------------
\section{Графы с нестандартной достижимостью}
%-----------------------------------------------------------------------------------------

\begin{frame}
\frametitle{Основные характеристики графов с нестандартной достижимостью}

	\begin{block}{Числовая характеристика произвольного пути}
		Числовой характеристикой произвольного пути $\mu$ называется отображение $\psi_\mu:N\to Z$, причём эта характеристика определяется рекуррентно по последней дуге пути с помощью некоторой заданной функции, a $\psi_\mu(0) = 0$.
	\end{block}
	
	\begin{block}{Формальное ограничение на достижимость}
		Выделяют два типа формальных ограничений:
		\begin{itemize}
			\item строгие --- следующая дуга пути обязана принадлежать конкретному подмножеству дуг графа;
			\item нестрогие --- следующая дуга пути обязана принадлежать этому подмножеству, только если среди инцидентных текущей вершине дуг есть хотя бы одна такая дуга.
		\end{itemize}
	
	\end{block}
\end{frame}

%-----------------------------------------------------------------------------------------

\begin{frame}
\frametitle{Пример глобального условия на достижимость}

	\begin{block}{Постановка условия}
		Рассмотрим орграф $G(X,U,f)$, в котором дуги разделены на два непересекающихся множества: стандартные $U_s$ и остаточные $U_o$. Формальное ограничение на достижимость: путь на графе $G$ допустимый, если в нем содержится количество остаточных дуг, кратное фиксированному целому числу $p$ ($|p| > 1$). 
	\end{block}

	Особенности:
	\begin{itemize}
		\item нет рекурсивности: ни сохранение, ни смена кратности в середине пути не имеет значения, важна лишь кратность в конце пути;
		\item префикс допустимого пути не обязательно допустим.
	\end{itemize}

\end{frame}

%-----------------------------------------------------------------------------------------

\begin{frame}
\frametitle{Сравнение локальных и глобальных условий на достижимость}
	
	\begin{table}
		\begin{tabular}{l | c | c }
			& Локальные условия    & Глобальные условия \\
			\hline \hline
			Рекурсивный            & Да, по правилу строгого    & Нет, не применимо \\
			выбор пути			   & /нестрогого условия	    & \\
			\hline \hline
			Модульность            & Да, префикс допустимого    & Нет, не применимо \\
			пути			       & пути всегда допустим       & \\
			\hline \hline
			Проверка               & Да, если путь не является  & Нет, любая часть, \\
			части пути			   & допустимым, можно 			& меньшая целого пути, \\
								   & выделить часть, которая    & не даст проверить \\
								   & также не является      	& условие на достижимость \\
			                       & допустимым путем
		\end{tabular}
	\end{table}
	
\end{frame}

%-----------------------------------------------------------------------------------------

\begin{frame}
\frametitle{Сведение задач на графах с глобальным условием на достижимость к задачам на классических графах}

	\begin{enumerate}
		\item предложен алгоритм построения вспомогательного графа;
		\item доказана теорема о соответствии любого пути на вспомогательном графе некоторому пути на исходном графе;
		\item задачи, сформулированные для исходного графа, можно решать классическими алгоритмами на вспомогательном графе: глобальные условия на достижимость учтены по построению.
	\end{enumerate}

\end{frame}

%-----------------------------------------------------------------------------------------
%-----------------------------------------------------------------------------------------
\section{Случайные процессы на графах с глобальными условиями на достижимость}
%-----------------------------------------------------------------------------------------

\begin{frame}\frametitle{Случайные процессы на графах с глобальными условиями на достижимость: формулировка задачи}

	\begin{itemize}
		\item орграф $G(X,U,f)$ --- сильно связный;
		
		\item $U_s = \emptyset$ и $U_o = U$ (частный случай);
		
		\item случайное блуждение частиц по графу задаётся матрицей переходов $P$;
		
		\item определено множество выходных вершин: из них можно попасть в отдельный от графа сток $s$, если путь частицы до выходной вершины является допустимым в соответствии с глобальным условием на кратность;
		
		\item требуется найти количество частиц, которые покинут граф в пределе, и распределение оставшихся частиц по вершинам.
	\end{itemize}

\end{frame}

%-----------------------------------------------------------------------------------------

\begin{frame}\frametitle{Методы решения}

	\begin{block}{Ищем множества $X_{in}$ и $X_{out}$}
	
		\begin{enumerate}
			\item «наивный»: если среди длин контуров графа и параметра $p$ есть взаимно простые числа, все частицы покинут граф, иначе можем выделить вершины, относящиеся к $X_{in}$;
			
			\item вычислительный: имея вспомогательный граф, строим его матрицу переходов, возводим в достаточно большую степень и анализируем вероятности;
			
			\item теоретический: анализируем компоненты сильной связности вспомогательного графа.
		\end{enumerate}
	
	\end{block}

\end{frame}

%-----------------------------------------------------------------------------------------
%-----------------------------------------------------------------------------------------
\section{Результаты}
%-----------------------------------------------------------------------------------------

\begin{frame}
\frametitle{Результаты}

	\begin{enumerate}
		\item рассмотрен пример глобального условия на достижимость --- условие проверки кратности;
		\item для условия проверки кратности описан алгоритм построения вспомогательного графа и доказана теорема о соответствии путей на исходном и вспомогательном графе;
		\item решена задача о предельном состоянии процесса случайного блуждания частиц на графе с условием проверки кратности.
	\end{enumerate} 

\end{frame}

%-----------------------------------------------------------------------------------------

\end{document}